%%%%%%%%%%%%%%%%%%%%%%%%%%%%%%%%%%%%%%%%%%%%%%%%%%%%%%%
% A template for Wiley article submissions.
% Developed by Overleaf. 
%
% Please note that whilst this template provides a 
% preview of the typeset manuscript for submission, it 
% will not necessarily be the final publication layout.
%
% Usage notes:
% The "blind" option will make anonymous all author, affiliation, correspondence and funding information.
% Use "num-refs" option for numerical citation and references style.
% Use "alpha-refs" option for author-year citation and references style.

\documentclass[alpha-refs]{wiley-article}
% \documentclass[blind,num-refs]{wiley-article}

% Add additional packages here if required
\usepackage{siunitx}

% Update article type if known
\papertype{Original Article}
% Include section in journal if known, otherwise delete
\paperfield{Journal Section}

\title{This is my title}

% List abbreviations here, if any. Please note that it is preferred that abbreviations be defined at the first instance they appear in the text, rather than creating an abbreviations list.
\abbrevs{ABC, a black cat; DEF, doesn't ever fret; GHI, goes home immediately.}

% Include full author names and degrees, when required by the journal.
% Use the \authfn to add symbols for additional footnotes and present addresses, if any. Usually start with 1 for notes about author contributions; then continuing with 2 etc if any author has a different present address.
\author[1\authfn{1}]{Author One PhD}
\author[2\authfn{1}]{Author A.~Two MD}
\author[2\authfn{2}]{Author Three PhD}
\author[2]{Author B.~Four}

\contrib[\authfn{1}]{Equally contributing authors.}

% Include full affiliation details for all authors
\affil[1]{Department, Institution, City, State or Province, Postal Code, Country}
\affil[2]{Department, Institution, City, State or Province, Postal Code, Country}

\corraddress{Author One PhD, Department, Institution, City, State or Province, Postal Code, Country}
\corremail{correspondingauthor@email.com}

\presentadd[\authfn{2}]{Department, Institution, City, State or Province, Postal Code, Country}

\fundinginfo{Funder One, Funder One Department, Grant/Award Number: 123456, 123457 and 123458; Funder Two, Funder Two Department, Grant/Award Number: 123459}

% Include the name of the author that should appear in the running header
\runningauthor{Author One et al.}

\begin{document}

\maketitle

\begin{abstract}
<<<<<<< HEAD
Systems engineering has developed a mature knowledge on how to design, integrate and manage complex industrial systems, whereas disciplines studying complex systems in nature or society also propose numerous tools for their understanding. Socio-technical systems, that situate at their intersection, could benefit from a higher integration between these. This position paper advocates for such integrated approaches. A bibliometric study through citation networks first illustrates the respective isolation of some of these approaches. We then produce a proof-of-concept of how the transfer of concepts from biology can be useful for the design of complex systems, in the particular case of transportation networks, using a biological network growth model to produce various optimal networks in terms of cost and efficiency. We finally discuss possible disciplinary positioning of such hybrid approaches.
\keywords{Systems Engineering; Complex Systems Science; Bibliometrics; Bio-inspired Network Design; Integrative Disciplines}
\end{abstract}




\cite{bar2003systems}


%%%%%%%%%%%%%%%%
\section{Introduction}
%%%%%%%%%%%%%%%%

% - socio-technical system at intersection of se/cs
% - different types of complexities : systems engineering / even within complex systems
% - but finally treat of the same objects ? debates of top-down/bottom-up


Socio-technical systems can be understood as the appropriation and use of technical artefacts by social agents, and lie therefore at the intersection of engineered systems and complex social systems. Urban systems are a typical illustration of such systems and of the related issues to design and manage them \cite{portugali2012complexity}. According to \cite{sheard2009principles}, concepts and methods originating from complex systems science, such as self-organization, chaos or emergence, should much more frequently be used for engineering that they currently are, since engineers would be missing fundamental properties of the systems they design. This paper aims at confirming this claim through the use of case studies. We first precise what is meant by \emph{systems engineering} and \emph{complex systems science}.


\paragraph{Systems engineering}

% - Incose architecting principles for systems of systems, early works
% - Model-based system engineering
% - international standards
% - current issues examples : coupling mbse and ple
The architecture of complex systems, in the sense of integrated systems of systems, is one objective of disciplines related to systems engineering. The International Council On Systems Engineering (INCOSE) is for example a major organization fostering the development of standardized practices and methodologies in that field, and gives regularly mature guidelines, such as for the architecture of systems of systems since a relatively long time \cite{maier1998architecting}. Model-based systems engineering (MBSE) \cite{estefan2007survey} is for example a methodology for designing systems, in which conceptual models of systems and data, together with their simulation, plays a central role. Several international standards have been introduced to structure practices \cite{schneider2013literature}. Current issues for the development of new methods include the coupling of existing methods with appropriated tools at large scales \cite{schafer2017challenges}.


\paragraph{Complex systems science}

There is not a single but several approaches to complexity in diverse fields of science, and we do not pretend to introduce a unified view of these. Indeed, \cite{chu2008criteria} recalls that several approaches to complexity do not currently converge to a single view. \cite{newman2011complex} gives an overview of concerned disciplines, that range from biology to physics and quantitative social science, and approaches, that include complex networks, agent-based modeling and simulation, non-linear dynamical systems. The common point of all the studied systems is to exhibit self-organization of a large number of elements, which translates into emerging properties at an upper level, in the sense of weak emergence which can be understood as the non-predictability of these properties, requiring simulation to understand the system \cite{bedau2002downward}. \cite{deffuant2015visions} actualize the metaphor of the Laplace Deamon, and develops three visions of complexity with progressive epistemological assumptions on the role of emergence, recalling that the computational complexity is already a barrier, but that higher orders of complexity, such as elementary constituting agents themselves complex, are often the rule in social systems.


\paragraph{Integrating complexities}

% - complex systems for se ; our paper is in that spirit, illustrating with concrete examples
% - similar with design thinking

Similarly to \cite{sheard2009principles}, \cite{ottino2004engineering} advocates for a stronger consideration of emerging properties in the engineering of complex systems, and claims for example that engineers and social scientists have much to exchange. \cite{jennings2003agent} suggests that agent-based systems are an interesting alternative for the design of control systems, in particular thanks to their increased flexibility and robustness. These issues of integrating complexities is indeed not particular to system engineering, as \cite{farmer2009economy} show that in the case of economics, policy-related benefits would be obtained by a more frequent use of agent-based approaches. In the case of systems engineering, the architectured artefacts directly form components of socio-technical systems, and the integration is thus directly relevant. \cite{durantin2017disruptive} propose that the integration of systems engineering and design thinking could allow a smoother design process.

The aim of this paper is to confirm these views through the illustration by case studies. More particularly, our contribution is twofold: (i) we proceed to a bibliometric analysis of some branches of systems engineering and complex systems, and show that their connection exist but is negligible regarding their internal connections; (ii) we develop a modeling example in which the generation of a biological network is used to design transportation systems, and therein give a proof-of-concept of the possible transfer of concepts and methods.


The rest of the paper is organized as follows: we first develop a bibliometric study, in order to illustrate through the exploration of citation networks the effective separation of some branches of system engineerings and of complex systems science. We then develop a modeling case study to give a proof-of-study of how complex systems concepts, in this case from biology, can be used for the design of systems. We finally discuss the disciplinary positioning implied by higher levels of integration.



%%%%%%%%%%%%%%%%
\section{A bibliometric insight}
%%%%%%%%%%%%%%%%

%\subsection{Context}

Statements about disciplines, their positioning and their relations, must often be taken with caution, including ours, as they will depend on the perspective taken to enter the problem, on the information available, on possible higher contexts implying sociological issues \cite{latour1977rhetorique}. They furthermore involve issues of reflexivity if they are done by researchers in the field themselves, implying to find what Morin calls a ``meta-viewpoint'' to construct an integrated knowledge \cite{edgar1986methode}. However, a growing body of knowledge in bibliometrics \cite{waltman2010unified}, that can be understood as a \emph{quantitative epistemology} \cite{chavalarias2013phylomemetic}, allows to construct maps of knowledge, using scientific citations networks or other proxies of knowledge such as patents \cite{bergeaud2017classifying}. We take here this approach to gain a quantitative insight into the relation between the disciplines we consider.

%\subsection{Method}
%  - criteria : survey ; enough citations.
%  - not too specialized, as network science ; agent-based ; dynamical systems ; etc.
% -> good compromise ?
% alternative

We use the tool and method provided by \cite{raimbault2017exploration} to reconstruct backward citation networks from open data. More precisely, given an initial corpus, the scientific papers citing this corpus referenced in google scholar are obtained at a given depth (i.e. including the papers citing the papers citing, and recursively). We start from two nodes, namely \cite{estefan2007survey} as the origin node for system engineerings (considering thus the subdomain of model-based systems engineering approaches) and \cite{newman2011complex} for complex systems (considering an entry from physics). These two references constitute a relevant initial corpus for the following reasons: (i) both are surveys of the literature aiming at giving an overview, and have received a significant number of citations (552 for \cite{estefan2007survey} and 157 for \cite{newman2011complex}) which are comparable in orders of magnitude; (ii) they are not too specific and should represent a consequent part of the fields (in comparison, for complex systems, an alternative such as \cite{stone2000multiagent} would be targeted on multi-agents systems). These choices have naturally an influence on the final results, and we therefore do not claim to construct full maps of the disciplines but to quantitatively illustrate our issue. Programs, data and results of these analyses are available on the open git repository of the project at \url{https://github.com/JusteRaimbault/ComplexitiesIntegration}.

=======
This is a generic template designed for use by multiple journals, which includes several options for customization. Please consult the author guidelines for the journal to which you are submitting in order to confirm that your manuscript will comply with the journal's requirements. Please replace this text with your abstract.
>>>>>>> 31ac023d4d109c2d20a0705e1bb20b1dd27fefe9

% Please include a maximum of seven keywords
\keywords{keyword 1, \emph{keyword 2}, keyword 3, keyword 4, keyword 5, keyword 6, keyword 7}
\end{abstract}

\section{First Level Heading}
Please lay out your article using the section headings and example objects below, and remember to delete all help text prior to submitting your article to the journal.

\begin{figure}[bt]
\centering
\includegraphics[width=6cm]{example-image-rectangle}
\caption{Although we encourage authors to send us the highest-quality figures possible, for peer-review purposes we are can accept a wide variety of formats, sizes, and resolutions. Legends should be concise but comprehensive – the figure and its legend must be understandable without reference to the text. Include definitions of any symbols used and define/explain all abbreviations and units of measurement.}
\end{figure}

\subsection{Second Level Heading}
If data, scripts or other artefacts used to generate the analyses presented in the article are available via a publicly available data repository, please include a reference to the location of the material within the article.

% Equations should be inserted using standard LaTeX equation and eqnarray environments, not as graphics, and should be set in the main text
This is an equation, numbered
\begin{equation}
\int_0^{+\infty}e^{-x^2}dx=\frac{\sqrt{\pi}}{2}
\end{equation}
And one that is not numbered
\begin{equation*}
e^{i\pi}=-1
\end{equation*}

\subsection{Adding Citations and a References List}

Please use a \verb|.bib| file to store your references. When using Overleaf to prepare your manuscript, you can upload a \verb|.bib| file or import your Mendeley, CiteULike or Zotero library directly as a \verb|.bib| file\footnote{see \url{https://www.overleaf.com/blog/184}}. You can then cite entries from it, like this: \cite{lees2010theoretical}. Just remember to specify a bibliography style, as well as the filename of the \verb|.bib|.

You can find a video tutorial here to learn more about BibTeX: \url{https://www.overleaf.com/help/97-how-to-include-a-bibliography-using-bibtex}.

This template provides two options for the citation and reference list style: 
\begin{description}
\item[Numerical style] Use \verb|\documentclass[...,num-refs]{wiley-article}|
\item[Author-year style] Use \verb|\documentclass[...,alpha-refs]{wiley-article}|
\end{description}

\subsubsection{Third Level Heading}
Supporting information will be included with the published article. For submission any supporting information should be supplied as separate files but referred to in the text.

Appendices will be published after the references. For submission they should be supplied as separate files but referred to in the text.

\paragraph{Fourth Level Heading}
% Here are examples of quotes and epigraphs.
\begin{quote}
The significant problems we have cannot be solved at the same level of thinking with which we created them.\endnote{Albert Einstein said this.}
\end{quote}

\begin{epigraph}{Albert Einstein}
Anyone who has never made a mistake has never tried anything new.
\end{epigraph}

\subparagraph{Fifth level heading}
Measurements should be given in SI or SI-derived units.
Chemical substances should be referred to by the generic name only. Trade names should not be used. Drugs should be referred to by their generic names. If proprietary drugs have been used in the study, refer to these by their generic name, mentioning the proprietary name, and the name and location of the manufacturer, in parentheses.

\begin{table}[bt]
\caption{This is a table. Tables should be self-contained and complement, but not duplicate, information contained in the text. They should be not be provided as images. Legends should be concise but comprehensive – the table, legend and footnotes must be understandable without reference to the text. All abbreviations must be defined in footnotes.}
\begin{threeparttable}
\begin{tabular}{lccrr}
\headrow
\thead{Variables} & \thead{JKL ($\boldsymbol{n=30}$)} & \thead{Control ($\boldsymbol{n=40}$)} & \thead{MN} & \thead{$\boldsymbol t$ (68)}\\
Age at testing & 38 & 58 & 504.48 & 58 ms\\
Age at testing & 38 & 58 & 504.48 & 58 ms\\
Age at testing & 38 & 58 & 504.48 & 58 ms\\
Age at testing & 38 & 58 & 504.48 & 58 ms\\
\hiderowcolors
stop alternating row colors from here onwards\\
Age at testing & 38 & 58 & 504.48 & 58 ms\\
Age at testing & 38 & 58 & 504.48 & 58 ms\\
\hline  % Please only put a hline at the end of the table
\end{tabular}

\begin{tablenotes}
\item JKL, just keep laughing; MN, merry noise.
\end{tablenotes}
\end{threeparttable}
\end{table}

\section*{acknowledgements}
Acknowledgements should include contributions from anyone who does not meet the criteria for authorship (for example, to recognize contributions from people who provided technical help, collation of data, writing assistance, acquisition of funding, or a department chairperson who provided general support), as well as any funding or other support information.

\section*{conflict of interest}
You may be asked to provide a conflict of interest statement during the submission process. Please check the journal's author guidelines for details on what to include in this section. Please ensure you liaise with all co-authors to confirm agreement with the final statement.

\printendnotes

% Submissions are not required to reflect the precise reference formatting of the journal (use of italics, bold etc.), however it is important that all key elements of each reference are included.
\bibliography{sample}

\begin{biography}[example-image-1x1]{A.~One}
Please check with the journal's author guidelines whether author biographies are required. They are usually only included for review-type articles, and typically require photos and brief biographies (up to 75 words) for each author.
\bigskip
\bigskip
\end{biography}

\graphicalabstract{example-image-1x1}{Please check the journal's author guildines for whether a graphical abstract, key points, new findings, or other items are required for display in the Table of Contents.}

\end{document}
